\documentclass{article}

\title{What are the Benefits of Open Source Software}
\author{Ahmed Shuaib - F2022-084}
\date{January 5th, 2023}

\usepackage{lmodern}
\usepackage[colorlinks=true]{hyperref}
\usepackage[margin=1.4in]{geometry}

\begin{document}

\maketitle
\pagebreak

\section*{What is open-source software?}
Open source software is computer software that is released under a license that allows the users to study, modify, and redistribute the source code and software. The source code is the set of instructions for the computer to execute, usually in the form of the syntax for a human-readable programming language like C, C++, or Java. This source code is then either compiled or directly executed after being parsed and translated. This means depending on the technology used, you may not even have a compiled executable for your open source application and instead just provide the user with the code and the required documentation to execute your application.

\subsection*{Licenses}
There are many variants of the open-source license. Apache License, GPL, GNU GPL, MIT, BSD, FreeBSD, MPL. Although all of these licenses have their differences, they all comply with the open source definition and are accepted by many organizations such as the
\href{https://opensource.org}{Open Source Initiative}, 
\href{https://www.fsf.org}{Free Software Foundation}, 
\href{https://foundation.mozilla.org}{Mozilla Foundation}, and 
\href{https://www.linuxfoundation.org}{Linux Foundation}.

\subsection*{Types of licenses}
\textbf{Premissive:}
\\
These licenses contain very few restrictions as to how software can be used, modified, and redistributed. Popular licenses include MIT, BSD, and Apache licenses.
\\\\
\textbf{Copyleft:}
\\
These licenses are restrictive, the software can be modified and redistributed as long as the redistributed software is free and open source (FOSS) and under the same license. Popular licenses include GPL, GNU GPL, AGPL, and LGPL. 
\\\\
\textbf{Proprietary:}
\\
These licenses are the most restrictive, usually used for commercial or proprietary software. The software cannot be modified, or redistributed, such software is not open source. Many popular software companies use such licenses as Adobe, Microsoft, Apple, Google, and Oracle.  
\\\\
\textbf{Public domain:}
\\
Every software needs a license, if one is not specified or the rights to the software have expired, the default public domain laws are applied, and such software can be used and modified by anyone. A good example of such software is the World Wide Web.

\pagebreak

\section*{What are the benefits of open-source?}
It's no secret, developing software is a really difficult task, being humans one of the ways we tackle difficult tasks is by working together. We use a lot of software on the daily bases that is highly complex in nature, the question pops up, how is this software developed? Usually, this large and complex software is the work of more than one person, usually a large team. Adobe, for example, one of the largest software companies in the world has about 25,988 employees as of 2021. Microsoft has about 221,000 (2022), and Google has 139,995 (2021). These giant organizations work on many large-scale software development projects. 
\\\\
How does Open-source software get developed? Open-source software encourages collaboration and better organization during the development phase of a project. Developers are more likely to add their ideas and features as well as improve already implemented features of an application. As anyone is allowed to view and manipulate the code of an application, bugs and exploits are found much faster and tend to be patched much faster. Anyone can use your application so you don't even need to have any programming knowledge to find bugs and help increase the accessibility of someone's application. Many open source applications are modular, which means that the backend, say all your application algorithms, functions, database systems can be separate from your graphical user interface, etc. This allows developers to work on core parts of their application without having to worry about implemeting other non essential parts of their application and leaving it to other developers to add on as plugins. 
\\\\
Many software projects use this technique including, VLC, KDE, Visual Studio Code, etc. Tools like git allow developers to not only have a centralized version control system but also manage several branches and forks of their applications while other developers improve their applications and your application without any conflicts. Once the change has been approved by the community, developers, etc. It can be merged into the main branch from which the other branches were originally made, giving consumers the most up-to-date and latest features while also being secure and fully transparent. I can go to Linus Torvalds Linux kernel repo and see what changes have been made today, what changes were made 2 months ago, who made them and why, \href{https://github.com/torvalds/linux}{here} go see for yourself. Now imagine that this is the same base kernel that has been forked many times and is now running in your android phone, TV, router/modem, and home security system. That is the benefit of open source.

\pagebreak

\section*{Discussions}
After talking to a few individuals about their thoughts on open-source software, It was quite evident that most people don't care much about the rules and restrictions that open-source software licenses have. Most people just care about the end-user experience, what does the software do, is it safe, does it provide the features required by me, does it have an intuitive and accessible user interface? While talking to other people, I realized why 75\% of desktop and laptop users use Microsoft Windows. The biggest reason is how familiar people are with the user interface. Windows has a user interface that is intuitive, easy to use, aesthetically pleasing, and powerful. Out of the many people I talked with about the benefits of open-source software, some were quite interested in the benefits for consumers like better security, new features, and the transparency that comes with open-source software.
\\\\
One of the students I talked to at the university cares a lot about data privacy, I explained to him how developers can study the source code and look for any bugs, or vulnerabilities that can result in a data breach. He mentioned to me how this is important for him, as he does not want his data and credentials to ever be compromised. He also believes our data should not be used in data algorithms to recommend us content and push advertisements to us. I mentioned to him that open-source platforms like Mastodon follow a different approach compared to social media platforms like Twitter and Facebook, these social media platforms are self-hosted and decentralized from a single entity. Operating systems like Ubuntu and Debian don’t collect data from our computers, as we use them. Instead, these open-source operating systems rely on the developers and users to work as a community to improve the software.
\\\\
Another individual I talked to wanted software to screen record tutorials for his clients, most of the software in this category is either paid or freeware software that has watermarks or locked features, I recommended him OBS (Open Broadcaster Software) an open-source broadcasting software used by many content creators around the world. This software has many new features added since its initial release in 2012. Over the years the software has become very feature rich compatible with almost all the newest hardware and supports many plugins, scripts, and modules to extend the features to your particular needs. This individual was happy to learn that the software is completely free and regularly gets updates that add new features and patches. Later on, he mentioned to me that he found the user interface for OBS to be powerful but quite complex for his needs, he was looking for something more simple to use, I introduced him to a fork of OBS known as Streamlabs OBS. Streamlabs has a much sleeker user interface and is a lot more simple to set up and get up and running while still using the powerful backend OBS provides.
\\\\\
Although Many people I talked to expressed their familiarity with using paid software like Windows, Photoshop, and Acrobat Reader. It is always nice to know that open-source software is there to fill in the gaps where such software is lacking in security, new features, and support. I encouraged all the people I talked with to try out open-source software whenever they face issues with the software they are currently using, they might just find it a lot better.
\end{document}

% Citations:
% https://en.wikipedia.org/wiki/Open-source_software
% https://en.wikipedia.org/wiki/Source_code
% https://opensource.org/osd
% https://opensource.org/licenses
% https://www.synopsys.com/blogs/software-security/5-types-of-software-licenses-you-need-to-understand/
% https://en.wikipedia.org/wiki/Permissive_software_license
% https://www.gnu.org/licenses/copyleft.en.html
% https://en.wikipedia.org/wiki/Google
% https://en.wikipedia.org/wiki/Microsoft
% https://en.wikipedia.org/wiki/Adobe_Inc.
% https://en.wikipedia.org/wiki/Streamlabs
% https://obsproject.com/wiki/
% https://guide.quickscrum.com/git-guide/


% MLA:
% “5 Types of Software Licenses You Need to Understand | Synopsys” Software Integrity Blog, 7 Oct. 2016, www.synopsys.com/blogs/software-security/5-types-of-software-licenses-you-need-to-understand/

% “Gnu.org” Www.gnu.org, www.gnu.org/licenses/copyleft.en.html

% “Licenses & Standards | Open Source Initiative” Opensource.org, 2019, opensource.org/licenses

% Open Source Initiative “The Open Source Definition | Open Source Initiative.” Opensource.org, 2007, opensource.org/osd

% “Permissive Software License” Wikipedia, 18 Mar. 2021, en.wikipedia.org/wiki/Permissive_software_license

% “Streamlabs” Wikipedia, 1 Jan. 2023, en.wikipedia.org/wiki/Streamlabs

% What Is Git? What Benefits Does Git Offer? guide.quickscrum.com/git-guide

% “Wiki - Wiki | OBS” Obsproject.com, obsproject.com/wiki/

% Wikipedia Contributors “Adobe Inc” Wikipedia, Wikimedia Foundation, 7 Dec. 2019, en.wikipedia.org/wiki/Adobe_Inc

% “Google” Wikipedia, Wikimedia Foundation, 18 Dec. 2018, en.wikipedia.org/wiki/Google

% “Microsoft” Wikipedia, Wikimedia Foundation, 25 Jan. 2019, en.wikipedia.org/wiki/Microsoft

% “Open-Source Software” Wikipedia, Wikimedia Foundation, 10 Nov. 2019, en.wikipedia.org/wiki/Open-source_software

% “Source Code” Wikipedia, Wikimedia Foundation, 1 Dec. 2019, en.wikipedia.org/wiki/Source_code